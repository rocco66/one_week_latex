\documentclass[10pt]{beamer}

\includeonly{tex/commands}

\usepackage{one_week_helloworld}

\setbeamertemplate{navigation symbols}{}
\setbeamersize{sidebar width left=1cm}
\usecolortheme{sidebartab}

\usetheme{Berkeley}

\title{Одна неделя с \LaTeX}
\author[Митрошин Максим]{Митрошин Максим \\ \texttt{\href{mailto:mitroshin@selectel.ru}{mitroshin@selectel.ru}}}
\date{\today}
\institute{Selectel}

\begin{document}

\begin{frame}
\titlepage
\end{frame}

\section{Синтаксис}
\begin{frame}[fragile]{Синтаксис}
  \begin{itemize}
    \item Text
      \begin{latexcode}
        Foo Bar Buz
      \end{latexcode}
    \pause  
    \item Command
      \begin{latexcode}
        \LaTeX
        \date{\today}
        \Large Waaagh!
      \end{latexcode}
    \pause  
    \item Environment
      \begin{latexcode}
        \begin{itemize}
          \item The first item
          \item The second item
        \end{itemize}
        {\Large Waaagh!}
      \end{latexcode}
    \pause  
    \item Comment
      \begin{latexcode}
        % TODO(Mitroshin): watch Firefly TV
      \end{latexcode}
  \end{itemize}
\end{frame}


\section{Основные команды}
\begin{frame}[fragile]{Основные команды}
  \begin{itemize}
    \item Класс документа (article, report, beamer)
      \begin{minted}{latex}
        \documentclass[options]{class}
      \end{minted}
    \pause  
    \item Используемые пакеты 
      \begin{minted}{latex}
        \usepackage{mathtools}
        \usepackage{minted}
        \usepackage[utf8x]{inputenc}
        \usepackage[russian]{babel}
      \end{minted}
    \pause  
    \item Опции, настройки, подключаемые файлы
      \begin{minted}{latex}
        \usetheme{Antibes}
        \includeonly{filename1,filename2,...}
      \end{minted}
    \pause  
    \item Документ
      \begin{minted}{latex}
        \begin{document}
          ...
        \end{document}
      \end{minted}
  \end{itemize}
\end{frame}

\section{Структура проекта}
\begin{frame}[fragile]{Структура проекта. Из Wiki}
  \begin{itemize}
    \item Корневой документ
      \begin{minted}{bash}
        ./document.tex
      \end{minted}
    \pause  
    \item Пакет проекта
      \begin{minted}{bash}
        ./mystyle.sty
      \end{minted}
    \pause  
    \item Документы для глав/секций/слайдов
      \begin{minted}{bash}
        ./tex/
      \end{minted}
    \pause  
    \item Изображения
      \begin{minted}{bash}
        ./img/
      \end{minted}
  \end{itemize}
\end{frame}


\section{Прикладные мелочи}

\begin{frame}[fragile]{Математические формулы}
  \begin{flushleft}
    \Large
    Процесс спадания звуковой энергии
    \begin{equation*}
      e(nt_{cp}) = \varepsilon_0 \exp\Bigg[ \frac{c_{3B} S ln(1 -
        \alpha)}{4V}t \Bigg]
    \end{equation*}
  \end{flushleft}
  \pause
  \begin{minted}{latex}
    \begin{equation*}
      e(nt_{cp}) = \varepsilon_0  * \exp
      \Bigg[
      \frac{c_{3B} * S * ln(1 -\alpha)} {4 * V} * t
      \Bigg]
    \end{equation*}
  \end{minted}
\end{frame}


\begin{frame}[fragile]{Форматирование текста}
  загибающийся текст
\end{frame}

\begin{frame}[fragile]{Создание своих команд}
  Отрисовка кода + beamer
\end{frame}

\section{Впечатления и выводы}

\begin{frame}{Раздражающие недостатки}
  баги при ошибках
  2(3?) раза выполнять одну и ту же команду? здравствуй bibtex 
  магия, много магии - нужно использовать один пакет чтобы не крашился другой 
  
  пример - overlay(beamer) + itemize!?
  
\end{frame}

\begin{frame}{Не осилил}
  \begin{itemize}
    \item Симпатичную анимацию
    \item Импорт файлов
  \end{itemize}  
\end{frame}

\begin{frame}{Emacs}
  сюда скриншотик
\end{frame}

\begin{frame}{Что использовалось в этой презентации}
  include - время компилляции  
\end{frame}

\begin{frame}{Вопросы?}
  Вопросы?
\end{frame}

\end{document}
